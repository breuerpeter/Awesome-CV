%!TEX TS-program = xelatex
%!TEX encoding = UTF-8 Unicode
% Awesome CV LaTeX Template for CV/Resume
%
% This template has been downloaded from:
% https://github.com/posquit0/Awesome-CV
%
% Author:
% Claud D. Park <posquit0.bj@gmail.com>
% http://www.posquit0.com
%
% Template license:
% CC BY-SA 4.0 (https://creativecommons.org/licenses/by-sa/4.0/)
%


%-------------------------------------------------------------------------------
% CONFIGURATIONS
%-------------------------------------------------------------------------------
% A4 paper size by default, use 'letterpaper' for US letter
\documentclass[11pt, a4paper]{awesome-cv}

% Configure page margins with geometry
\geometry{left=1.4cm, top=1.4cm, right=1.4cm, bottom=1.8cm, footskip=.5cm}

% Specify the location of the included fonts
\fontdir[fonts/]

% Color for highlights
% Awesome Colors: awesome-emerald, awesome-skyblue, awesome-red, awesome-pink, awesome-orange
%                 awesome-nephritis, awesome-concrete, awesome-darknight

% Uncomment if you would like to specify your own color
% \definecolor{awesome}{HTML}{CA63A8}

% Colors for text
% Uncomment if you would like to specify your own color
% \definecolor{darktext}{HTML}{414141}
% \definecolor{text}{HTML}{333333}
% \definecolor{graytext}{HTML}{5D5D5D}
% \definecolor{lighttext}{HTML}{999999}

% Set false if you don't want to highlight section with awesome color
\setbool{acvSectionColorHighlight}{true}

% If you would like to change the social information separator from a pipe (|) to something else
\renewcommand{\acvHeaderSocialSep}{\quad\textbar\quad}


%-------------------------------------------------------------------------------
%	PERSONAL INFORMATION
%	Comment any of the lines below if they are not required
%-------------------------------------------------------------------------------
% Available options: circle|rectangle,edge/noedge,left/right
 \photo[circle,noedge,left]{./portrait-compressed.png}
\name{Peter}{Breuer}
%\position{Upcoming Master Student}%{\enskip\cdotp\enskip}Program}
\address{Lerchenrain 13, 8046 Zurich, Switzerland}

\mobile{+41 76 470 25 50}
\email{pbreuer@ethz.ch}
\homepage{pbreuer.de}
%\github{mygithub}
\linkedin{linkedin.com/in/peter-breuer}
% \gitlab{my-gitlab-id}
% \stackoverflow{SO-id}{SO-name}
%\twitter{@twitterhandle}
% \skype{skype-id}
% \reddit{reddit-id}
 \extrainfo{21.01.2000 (Leipzig, Germany)}

%-------------------------------------------------------------------------------
\begin{document}

% Print the header with above personal informations
% Give optional argument to change alignment(C: center, L: left, R: right)
\makecvheader[C]

% Print the footer with 3 arguments(<left>, <center>, <right>)
% Leave any of these blank if they are not needed
\makecvfooter
  {February 2024}
  {Peter Breuer ~~~·~~~ Resume}
  {\thepage}


%-------------------------------------------------------------------------------
%	CV/RESUME CONTENT
%	Each section is imported separately, open each file in turn to modify content
%-------------------------------------------------------------------------------
%%-------------------------------------------------------------------------------
%	SECTION TITLE
%-------------------------------------------------------------------------------
\cvsection{Summary}


%-------------------------------------------------------------------------------
%	CONTENT
%-------------------------------------------------------------------------------
\begin{cvparagraph}

%---------------------------------------------------------
Master's student in robotics, systems and control at ETH Zurich (graduation in Fall 2024). Experience in legged and aerial robotics (sUAS). Primary interest in GNC software engineering for autonomous UAVs. Knowledge in robotics subdomains including computer vision, kinematics and dynamics, path planning, state estimation, model predictive control, and reinforcement learning (neural control policies).

\end{cvparagraph}

%%-------------------------------------------------------------------------------
%	SECTION TITLE
%-------------------------------------------------------------------------------
\cvsection{Work Experience}


%-------------------------------------------------------------------------------
%	CONTENT
%-------------------------------------------------------------------------------
\begin{cventries}

%---------------------------------------------------------
  \cventry
    {Research Assistant at the Robotic Systems Lab (RSL)} % Job title
    {ETH Zurich} % Organization
    {Zurich, Switzerland} % Location
    {09.2022 - 12.2023} % Date(s)
    {
      \begin{cvitems}
        \item Developed a GUI with RQt for controlling a quadrupedal robot using ROS 2
        \item Set up a containerized ROS 2 development environment using Docker
      \end{cvitems}
    }

  \cventry
    {Research Assistant at the Autonomous Systems Lab (ASL)} % Job title
    {} % Organization
    {} % Location
    {} % Date(s)
    {
      \begin{cvitems}
        \item Rapid prototyping of a payload retrieval mechanism for a VTOL UAV using additive manufacturing
      \end{cvitems}
    }

%---------------------------------------------------------
  \cventry
    {Robotics Software Engineering Intern} % Job title
    {Freefly Systems Inc.} % Organization
    {Seattle metropolitan area, USA} % Location
    {03.2022 - 07.2022} % Date(s)
    {
      \begin{cvitems}
        \item Created a PX4 flight mode for the \href{https://freeflysystems.com/astro}{industrial drone \textit{<<Astro>>} \ExternalLink} enabling precise framing and tracking with a gimbaled camera payload
        \item Added the possibility to configure a MAVLink-enabled gimbal in the ground control station
        \item Created an onboard application with pymavlink to request and publish received distance sensor readings via MAVLink messages
        \item Proof of concept for an app using MAVSDK to shoot hyperlapses with MAVLink-enabled UAVs
      \end{cvitems}
    }

%---------------------------------------------------------
\cventry
  {Teaching Assistant at the Engineering Design and Computing Laboratory (EDAC)} % Job title
  {ETH Zurich} % Organization
  {Zurich, Switzerland} % Location
  {09.2019 - 12.2021} % Date(s)
  {
    \begin{cvitems}
      \item Course: Technical Drawing and CAD
    \end{cvitems}
  }

  \cventry
  {Teaching Assistant at the Institute for Dynamic Systems and Control (IDSC)} % Job title
  {} % Organization
  {} % Location
  {} % Date(s)
  {
    \begin{cvitems}
      \item Course: Control Systems I and II
    \end{cvitems}
  }

%---------------------------------------------------------
\end{cventries}

%%-------------------------------------------------------------------------------
%	SECTION TITLE
%-------------------------------------------------------------------------------
\cvsection{Honors \& Awards}


%-------------------------------------------------------------------------------
%	SUBSECTION TITLE
%-------------------------------------------------------------------------------
\cvsubsection{Awards}


%-------------------------------------------------------------------------------
%	CONTENT
%-------------------------------------------------------------------------------
\begin{cvhonors}

%---------------------------------------------------------
  \cvhonor
    {Duke of Edinburgh's International Silver Award} % Award
    {Duke of Edinburgh's Award} % Event
    {Leipzig, Germany} % Location
    {06/2018} % Date(s)

%---------------------------------------------------------
  \cvhonor
    {Outstanding Cambridge Learner Awards} % Award
    {Cambridge Assessment International Education} % Event
    {Leipzig, Germany} % Location
    {10/2016} % Date(s)

%---------------------------------------------------------
\end{cvhonors}
%%-------------------------------------------------------------------------------
%	SECTION TITLE
%-------------------------------------------------------------------------------
\cvsection{Presentation}


%-------------------------------------------------------------------------------
%	CONTENT
%-------------------------------------------------------------------------------
\begin{cventries}

%---------------------------------------------------------
  \cventry
    {Presenter for <Hosting Web Application for Free utilizing GitHub, Netlify and CloudFlare>} % Role
    {DevFest Seoul by Google Developer Group Korea} % Event
    {Seoul, S.Korea} % Location
    {Nov. 2017} % Date(s)
    {
      \begin{cvitems} % Description(s)
        \item {Introduced the history of web technology and the JAM stack which is for the modern web application development.}
        \item {Introduced how to freely host the web application with high performance utilizing global CDN services.}
      \end{cvitems}
    }

%---------------------------------------------------------
  \cventry
    {Presenter for <DEFCON 20th : The way to go to Las Vegas>} % Role
    {6th CodeEngn (Reverse Engineering Conference)} % Event
    {Seoul, S.Korea} % Location
    {Jul. 2012} % Date(s)
    {
      \begin{cvitems} % Description(s)
        \item {Introduced CTF(Capture the Flag) hacking competition and advanced techniques and strategy for CTF}
      \end{cvitems}
    }

%---------------------------------------------------------
\end{cventries}

%%-------------------------------------------------------------------------------
%	SECTION TITLE
%-------------------------------------------------------------------------------
\cvsection{Writing}


%-------------------------------------------------------------------------------
%	CONTENT
%-------------------------------------------------------------------------------
\begin{cventries}

%---------------------------------------------------------
  \cventry
    {Founder \& Writer} % Role
    {A Guide for Developers in Start-up} % Title
    {Facebook Page} % Location
    {Jan. 2015 - PRESENT} % Date(s)
    {
      \begin{cvitems} % Description(s)
        \item {Drafted daily news for developers in Korea about IT technologies, issues about start-up.}
      \end{cvitems}
    }

%---------------------------------------------------------
\end{cventries}

%%-------------------------------------------------------------------------------
%	SECTION TITLE
%-------------------------------------------------------------------------------
\cvsection{Program Committees}


%-------------------------------------------------------------------------------
%	CONTENT
%-------------------------------------------------------------------------------
\begin{cvhonors}

%---------------------------------------------------------
  \cvhonor
    {Problem Writer} % Position
    {2016 CODEGATE Hacking Competition World Final} % Committee
    {S.Korea} % Location
    {2016} % Date(s)

%---------------------------------------------------------
  \cvhonor
    {Organizer \& Co-director} % Position
    {1st POSTECH Hackathon} % Committee
    {S.Korea} % Location
    {2013} % Date(s)

%---------------------------------------------------------
\end{cvhonors}

%%-------------------------------------------------------------------------------
%	SECTION TITLE
%-------------------------------------------------------------------------------
\cvsection{Education}


%-------------------------------------------------------------------------------
%	CONTENT
%-------------------------------------------------------------------------------
\begin{cventries}

%---------------------------------------------------------
  \cventry
    {MSc Robotics, Systems and Control} % Degree
    {ETH Zurich} % Institution
    {Zurich, Switzerland} % Location
    {09/2022 - present} % Date(s)
    {
      \begin{cvitems} % Description(s) bullet points
        \item Semester project: \textit{Altitude Estimation for UAV Operations Over Water - Integration and Testing of LiDAR and
        Radar Distance Sensors} under the supervision of Dr. Guillaume Ducard at the Institute for Dynamic Systems and Control (IDSC) group of Prof. Onder
        \item Master's thesis: \textit{Autonomous Tracking of Moving Subjects With UAVs Using Deep Reinforcement Learning} under the supervision of Jiaxu Xing and Angel Romero at the Robotics and Perception Group (University of Zurich) led by Prof. Davide Scaramuzza
      \end{cvitems}
    }

%---------------------------------------------------------
  \cventry
  {MSc Semester Exchange Program} % Degree
  {National University of Singapore} % Institution
  {Singapore} % Location
  {01/2023 - 05/2023} % Date(s)
  {
  }

%---------------------------------------------------------
  \cventry
    {BSc Mechanical Engineering} % Degree
    {ETH Zurich} % Institution
    {Zurich, Switzerland} % Location
    {09/2018 - 08/2022} % Date(s)
    {
      \begin{cvitems} % Description(s) bullet points
        \item GPA: 5.5/6.0 (approx. top 4\% of 207)
        \item Student project: \textit{Dyana - Dynamic Quadrupedal Animatronic} - A quadrupedal robot developed by an interdisciplinary team of 14 students from multiple Swiss universities. Dyana is capable of dynamic motions and conveys a unique, life‑like impression through its feline appearance. The project was hosted by the Robotics System Lab (RSL) led by Prof. Marco Hutter and was supported by several industry sponsors. More info from the \href{https://www.youtube.com/watch?v=--waMaVgZAg}{roll-out presentation (English subtitles) \ExternalLink}.
        \item Bachelor's thesis: \textit{Low‑ and High‑Level Control for Testing Single Legs of the Quadrupedal Animatronic Dyana} in collaboration with Marco Trentini and under the supervision of Fabian Tischhauser and Marcus Montenegro at the RSL. We developed a ROS 1 (C++) framework for dynamic testing of robotic legs using virtual model- and inverse dynamics control. It allowed us to test the mechanical integrity of the front and hind legs of \textit{Dyana} in a wall-mounted linear guide rail setup and evaluate the commercially available \textit{T-MOTOR AK10-9} actuators used.
      \end{cvitems}
    }

%---------------------------------------------------------
  \cventry
  {International Baccalaureate (IB) Billingual Diploma} % Degree
  {Leipzig International School} % Institution
  {Leipzig, Germany} % Location
  {08/2016 - 05/2018} % Date(s)
  {
    \begin{cvitems}
      \item Score: 42/45 (top 4.54\% worldwide)
      \item Higher level subjects: Mathematics, Physics, English (B)
      \item Standard level subjects: Chemistry, Geography, German (A)
      \item Valedictorian
    \end{cvitems}
  }

%---------------------------------------------------------
  \cventry
  {International General Certificate of Secondary Education (IGCSE)}
  {Leipzig International School}
  {Leipzig, Germany}
  {08/2014 - 06/2016}
  {
    \begin{cvitems}
      \item {Top of the year group}
    \end{cvitems}
  }
%---------------------------------------------------------
\end{cventries}

%%-------------------------------------------------------------------------------
%	SECTION TITLE
%-------------------------------------------------------------------------------
\cvsection{Extracurricular Activity}


%-------------------------------------------------------------------------------
%	CONTENT
%-------------------------------------------------------------------------------
\begin{cventries}

%---------------------------------------------------------
  \cventry
    {Core Member \& President at 2013} % Affiliation/role
    {PoApper (Developers' Network of POSTECH)} % Organization/group
    {Pohang, S.Korea} % Location
    {Jun. 2010 - Jun. 2017} % Date(s)
    {
      \begin{cvitems} % Description(s) of experience/contributions/knowledge
        \item {Reformed the society focusing on software engineering and building network on and off campus.}
        \item {Proposed various marketing and network activities to raise awareness.}
      \end{cvitems}
    }

%---------------------------------------------------------
  \cventry
    {Member} % Affiliation/role
    {PLUS (Laboratory for UNIX Security in POSTECH)} % Organization/group
    {Pohang, S.Korea} % Location
    {Sep. 2010 - Oct. 2011} % Date(s)
    {
      \begin{cvitems} % Description(s) of experience/contributions/knowledge
        \item {Gained expertise in hacking \& security areas, especially about internal of operating system based on UNIX and several exploit techniques.}
        \item {Participated on several hacking competition and won a good award.}
        \item {Conducted periodic security checks on overall IT system as a member of POSTECH CERT.}
        \item {Conducted penetration testing commissioned by national agency and corporation.}
      \end{cvitems}
    }

%---------------------------------------------------------
\end{cventries}


%-------------------------------------------------------------------------------
%	CV/RESUME CONTENT
%	Each section is imported separately, open each file in turn to modify content
%   Comment out any sections that are not appropriate for your CV
%-------------------------------------------------------------------------------
\cvsection{About}
\vspace{2mm}

\begin{cvparagraph}
    Master's student in robotics, systems and control at ETH Zurich (graduation in Fall 2024). Experience in the field of legged robotics and sUAS (PX4 autopilot). Particularly interested in GNC software engineering for autonomous UAVs. Knowledge of a wide spectrum of robotics subdomains including computer vision (SLAM), dynamic modeling, model predictive control, path planning, state estimation and machine learning.
\end{cvparagraph}

%-------------------------------------------------------------------------------
%	SECTION TITLE
%-------------------------------------------------------------------------------
\cvsection{Education}


%-------------------------------------------------------------------------------
%	CONTENT
%-------------------------------------------------------------------------------
\begin{cventries}

%---------------------------------------------------------
  \cventry
    {MSc Robotics, Systems and Control} % Degree
    {ETH Zurich} % Institution
    {Zurich, Switzerland} % Location
    {09/2022 - present} % Date(s)
    {
      \begin{cvitems} % Description(s) bullet points
        \item Semester project: \textit{Altitude Estimation for UAV Operations Over Water - Integration and Testing of LiDAR and
        Radar Distance Sensors} under the supervision of Dr. Guillaume Ducard at the Institute for Dynamic Systems and Control (IDSC) group of Prof. Onder
        \item Master's thesis: \textit{Autonomous Tracking of Moving Subjects With UAVs Using Deep Reinforcement Learning} under the supervision of Jiaxu Xing and Angel Romero at the Robotics and Perception Group (University of Zurich) led by Prof. Davide Scaramuzza
      \end{cvitems}
    }

%---------------------------------------------------------
  \cventry
  {MSc Semester Exchange Program} % Degree
  {National University of Singapore} % Institution
  {Singapore} % Location
  {01/2023 - 05/2023} % Date(s)
  {
  }

%---------------------------------------------------------
  \cventry
    {BSc Mechanical Engineering} % Degree
    {ETH Zurich} % Institution
    {Zurich, Switzerland} % Location
    {09/2018 - 08/2022} % Date(s)
    {
      \begin{cvitems} % Description(s) bullet points
        \item GPA: 5.5/6.0 (approx. top 4\% of 207)
        \item Student project: \textit{Dyana - Dynamic Quadrupedal Animatronic} - A quadrupedal robot developed by an interdisciplinary team of 14 students from multiple Swiss universities. Dyana is capable of dynamic motions and conveys a unique, life‑like impression through its feline appearance. The project was hosted by the Robotics System Lab (RSL) led by Prof. Marco Hutter and was supported by several industry sponsors. More info from the \href{https://www.youtube.com/watch?v=--waMaVgZAg}{roll-out presentation (English subtitles) \ExternalLink}.
        \item Bachelor's thesis: \textit{Low‑ and High‑Level Control for Testing Single Legs of the Quadrupedal Animatronic Dyana} in collaboration with Marco Trentini and under the supervision of Fabian Tischhauser and Marcus Montenegro at the RSL. We developed a ROS 1 (C++) framework for dynamic testing of robotic legs using virtual model- and inverse dynamics control. It allowed us to test the mechanical integrity of the front and hind legs of \textit{Dyana} in a wall-mounted linear guide rail setup and evaluate the commercially available \textit{T-MOTOR AK10-9} actuators used.
      \end{cvitems}
    }

%---------------------------------------------------------
  \cventry
  {International Baccalaureate (IB) Billingual Diploma} % Degree
  {Leipzig International School} % Institution
  {Leipzig, Germany} % Location
  {08/2016 - 05/2018} % Date(s)
  {
    \begin{cvitems}
      \item Score: 42/45 (top 4.54\% worldwide)
      \item Higher level subjects: Mathematics, Physics, English (B)
      \item Standard level subjects: Chemistry, Geography, German (A)
      \item Valedictorian
    \end{cvitems}
  }

%---------------------------------------------------------
  \cventry
  {International General Certificate of Secondary Education (IGCSE)}
  {Leipzig International School}
  {Leipzig, Germany}
  {08/2014 - 06/2016}
  {
    \begin{cvitems}
      \item {Top of the year group}
    \end{cvitems}
  }
%---------------------------------------------------------
\end{cventries}

\newpage
\cvsection{Professional Experience}
\vspace{-2.5mm}  

\begin{cvskills}

    \cventry
    {Research Assistant}
    {ETH Zurich}
    {Zurich, Switzerland}
    {09.2022 - 12.2023}
    {
    \begin{cvitems}
        \item \textbf{Robotic Systems Lab (RSL)}
        \begin{itemize}
        \item Developing a GUI for controlling a quadrupedal robot using ROS 2 (RQt)
        \item Set up containerized ROS 2 development environment (Docker)
        \end{itemize}
        \item \textbf{Autonomous Systems Lab (ASL)}
        \begin{itemize}
        \item Rapid prototyping of a payload retrieval mechanism for a VTOL UAV
        \end{itemize}
    \end{cvitems}
    }

    \cventry
    {Robotics Engineering Intern}
    {Freefly Systems Inc.}
    {Woodinville, USA}
    {03.2022 - 07.2022}
    {\begin{cvitems}
        \item Software development for the \href{https://freeflysystems.com/astro}{industrial drone \textit{<<Astro>>} \ExternalLink}
        \begin{itemize}
            \item Created PX4 flight mode enabling precise framing and tracking with gimbaled camera payload
            \item Added possibility to configure MAVLink-enabled gimbal in ground control station
            \item Created onboard application to monitor distance sensor readings in ground control station
            \item Proof of concept: offboard app using MAVSDK to create hyperlapses with MAVLink-enabled UAVs
        \end{itemize}
    \end{cvitems}
    }

    \cventry
    {Teaching Assistant}
    {ETH Zurich}
    {Zurich, Switzerland}
    {09.2019 - 12.2021}
    {
    \begin{cvitems}
        \item \textbf{Engineering Design and Computing Laboratory (EDAC)}
        \begin{itemize}
            \item Course: Technical Drawing and CAD
        \end{itemize}
        \item \textbf{Institute for Dynamic Systems and Control (IDSC)}
        \begin{itemize}
            \item Courses: Control Systems I, Control Systems II
        \end{itemize}
    \end{cvitems}
    }

    \cventry
    {Creative Services}
    {Misc}
    {Leipzig, Germany}
    {02.2015 - 09.2017} % Year
    {\begin{cvitems}
        \item Website Development | FST - Fenster Solar Technik GmbH
        \item Event Photography (Abschlussveranstaltung) | Arwed Rossbach Schule
        \item Advertisement Photography | pioneer communications GmbH
        \item Product and Interior Photography | Webervogel GmbH
    \end{cvitems}
    }
    
\end{cvskills}
%\cvsection{Publications}

\cvsubsection{Published}

 \begin{cvpubs}
     \cvpub{\textbf{Name, Y}, Author Two, Author Three, Author Four, Author Five. 2017. Title of our fancy paper, which has a very long name to demonstrate what happens when this goes onto two lines. Title of the Journal, 2(1): 1000-1100.}
     \cvpub{\textbf{Name, Y}, Author Two, Author Three. 2016. Title of our other fancy paper, which also has a very long name so that the citation goes onto the next line. Journal, 1(1): 1000-1100.}
 \end{cvpubs}

\cvsubsection{In Review}

\begin{cvpubs}
    \cvpub{\textbf{M. Dubied, M. Y. Michelis, A. Spielberg, R. K.  Katzschmann}. 2021. Sim-to-Real for Differentiable Projective Dynamics on Soft Robots: Meshing, Damping, and Actuation. IEEE Robotics and Automation Letters (RA-L)}
    \cvpub{Manuscript 2}
\end{cvpubs}

\cvsubsection{In Prep}

\begin{cvpubs}
    \small \color{black}
    \cvpub{Manuscript 1}
    \cvpub{Manuscript 2}
\end{cvpubs}
\newpage
%\cvsection{Research Experience}

\begin{cventries}
    \cventry
        {Advisors: Dr. Guillaume J.J. Ducard, Prof. Dr. Christopher Onder}
        {ETH Zurich - Institute for Dynamic Systems and Control (IDSC)}
        {Zurich - Switzerland}
        {11.2023 - present}
        {
        \begin{cvitems}
            \item Semester project: enabling autonomous landing (and takeoff) of an amphibious fixed wing RC aircraft on water
            \begin{itemize}
                \item Distance sensor (LiDAR, mmWave radar) testing, evaluation and integration
                \item Hardware setup (Pixhawk FMUv6X with Raspberry Pi CM4 and sensors)
                \item PX4 firmware setup (in progress)
                \item PX4 SIH simulation (in progress)
                \item Flight testing (in progress)
            \end{itemize}
        \end{cvitems}
    }
\end{cventries}
%-------------------------------------------------------------------------------
%	SECTION TITLE
%-------------------------------------------------------------------------------
\cvsection{Honors \& Awards}


%-------------------------------------------------------------------------------
%	SUBSECTION TITLE
%-------------------------------------------------------------------------------
\cvsubsection{International Awards}


%-------------------------------------------------------------------------------
%	CONTENT
%-------------------------------------------------------------------------------
\begin{cvhonors}

%---------------------------------------------------------
  \cvhonor
    {2nd Place} % Award
    {AWS ASEAN AI/ML GameDay} % Event
    {Online} % Location
    {2021} % Date(s)

%---------------------------------------------------------
  \cvhonor
    {Finalist} % Award
    {DEFCON 28th CTF Hacking Competition World Final} % Event
    {Las Vegas, U.S.A} % Location
    {2020} % Date(s)

%---------------------------------------------------------
  \cvhonor
    {Finalist} % Award
    {DEFCON 26th CTF Hacking Competition World Final} % Event
    {Las Vegas, U.S.A} % Location
    {2018} % Date(s)

%---------------------------------------------------------
  \cvhonor
    {Finalist} % Award
    {DEFCON 25th CTF Hacking Competition World Final} % Event
    {Las Vegas, U.S.A} % Location
    {2017} % Date(s)

%---------------------------------------------------------
  \cvhonor
    {Finalist} % Award
    {DEFCON 22nd CTF Hacking Competition World Final} % Event
    {Las Vegas, U.S.A} % Location
    {2014} % Date(s)

%---------------------------------------------------------
  \cvhonor
    {Finalist} % Award
    {DEFCON 21st CTF Hacking Competition World Final} % Event
    {Las Vegas, U.S.A} % Location
    {2013} % Date(s)

%---------------------------------------------------------
  \cvhonor
    {Finalist} % Award
    {DEFCON 19th CTF Hacking Competition World Final} % Event
    {Las Vegas, U.S.A} % Location
    {2011} % Date(s)

%---------------------------------------------------------
  \cvhonor
    {6th Place} % Award
    {SECUINSIDE Hacking Competition World Final} % Event
    {Seoul, S.Korea} % Location
    {2012} % Date(s)

%---------------------------------------------------------
\end{cvhonors}


%-------------------------------------------------------------------------------
%	SUBSECTION TITLE
%-------------------------------------------------------------------------------
\cvsubsection{Domestic Awards}


%-------------------------------------------------------------------------------
%	CONTENT
%-------------------------------------------------------------------------------
\begin{cvhonors}

%---------------------------------------------------------
  \cvhonor
    {2nd Place} % Award
    {AWS Korea GameDay} % Event
    {Seoul, S.Korea} % Location
    {2021} % Date(s)

%---------------------------------------------------------
  \cvhonor
    {3rd Place} % Award
    {WITHCON Hacking Competition Final} % Event
    {Seoul, S.Korea} % Location
    {2015} % Date(s)

%---------------------------------------------------------
  \cvhonor
    {Silver Prize} % Award
    {KISA HDCON Hacking Competition Final} % Event
    {Seoul, S.Korea} % Location
    {2017} % Date(s)

%---------------------------------------------------------
  \cvhonor
    {Silver Prize} % Award
    {KISA HDCON Hacking Competition Final} % Event
    {Seoul, S.Korea} % Location
    {2013} % Date(s)

%---------------------------------------------------------
  \cvhonor
    {2nd Award} % Award
    {HUST Hacking Festival} % Event
    {S.Korea} % Location
    {2013} % Date(s)

%---------------------------------------------------------
  \cvhonor
    {3rd Award} % Award
    {HUST Hacking Festival} % Event
    {S.Korea} % Location
    {2010} % Date(s)

%---------------------------------------------------------
  \cvhonor
    {3rd Award} % Award
    {Holyshield 3rd Hacking Festival} % Event
    {S.Korea} % Location
    {2012} % Date(s)

%---------------------------------------------------------
  \cvhonor
    {2nd Award} % Award
    {Holyshield 3rd Hacking Festival} % Event
    {S.Korea} % Location
    {2011} % Date(s)

%---------------------------------------------------------
  \cvhonor
    {5th Place} % Award
    {PADOCON Hacking Competition Final} % Event
    {Seoul, S.Korea} % Location
    {2011} % Date(s)

%---------------------------------------------------------
\end{cvhonors}

%-------------------------------------------------------------------------------
%	SUBSECTION TITLE
%-------------------------------------------------------------------------------
\cvsubsection{Community}


%-------------------------------------------------------------------------------
%	CONTENT
%-------------------------------------------------------------------------------
\begin{cvhonors}

%---------------------------------------------------------
  \cvhonor
    {AWS Community Builder (Container)} % Award
    {Amazon Web Services (AWS)} % Event
    {} % Location
    {2022} % Date(s)

%---------------------------------------------------------
  \cvhonor
    {HashiCorp Ambassador} % Award
    {HashiCorp} % Event
    {} % Location
    {2022} % Date(s)

%---------------------------------------------------------
\end{cvhonors}

%%-------------------------------------------------------------------------------
%	SECTION TITLE
%-------------------------------------------------------------------------------
\cvsection{Presentation}


%-------------------------------------------------------------------------------
%	CONTENT
%-------------------------------------------------------------------------------
\begin{cventries}

%---------------------------------------------------------
  \cventry
    {Presenter for <Hosting Web Application for Free utilizing GitHub, Netlify and CloudFlare>} % Role
    {DevFest Seoul by Google Developer Group Korea} % Event
    {Seoul, S.Korea} % Location
    {Nov. 2017} % Date(s)
    {
      \begin{cvitems} % Description(s)
        \item {Introduced the history of web technology and the JAM stack which is for the modern web application development.}
        \item {Introduced how to freely host the web application with high performance utilizing global CDN services.}
      \end{cvitems}
    }

%---------------------------------------------------------
  \cventry
    {Presenter for <DEFCON 20th : The way to go to Las Vegas>} % Role
    {6th CodeEngn (Reverse Engineering Conference)} % Event
    {Seoul, S.Korea} % Location
    {Jul. 2012} % Date(s)
    {
      \begin{cvitems} % Description(s)
        \item {Introduced CTF(Capture the Flag) hacking competition and advanced techniques and strategy for CTF}
      \end{cvitems}
    }

%---------------------------------------------------------
  \cventry
    {Presenter for <Metasploit 101>} % Role
    {6th Hacking Camp - S.Korea} % Event
    {S.Korea} % Location
    {Sep. 2012} % Date(s)
    {
      \begin{cvitems} % Description(s)
        \item {Introduced basic procedure for penetration testing and how to use Metasploit}
      \end{cvitems}
    }

%---------------------------------------------------------
\end{cventries}

%%-------------------------------------------------------------------------------
%	SECTION TITLE
%-------------------------------------------------------------------------------
\cvsection{Teaching Experience}


%-------------------------------------------------------------------------------
%	CONTENT
%-------------------------------------------------------------------------------

\begin{cvhonors}

%---------------------------------------------------------
  \cvhonor
    {Course} % Course
    {Teaching Assistant} % Course
    {} % Location
    {Spring 2018} % Date(s)

%---------------------------------------------------------
  \cvhonor
    {Course} % Course
    {Teaching Assistant} % Position
    {} % Location
    {Spring 2017} % Date(s)

%---------------------------------------------------------
  \cvhonor
    {Course} % Course
    {Teaching Assistant} % Position
    {} % Location
    {Fall 2016} % Date(s)

%---------------------------------------------------------
  \cvhonor
    {Course} % Course
    {Teaching Assistant} % Position
    {} % Location
    {Fall 2015} % Date(s)

%---------------------------------------------------------
  \cvhonor
    {Teaching Fellow Position} % Course
    {University Department} % Committee
    {} % Location
    {Spring 2015} % Date(s)

%---------------------------------------------------------
  \cvhonor
    {Course} % Course
    {Teaching Assistant} % Committee
    {} % Location
    {Fall 2014} % Date(s)

%---------------------------------------------------------
\end{cvhonors}

%\cvsection{Mentoring}

\begin{cvhonors}

%---------------------------------------------------------
  \cvhonor
    {Undergraduate Name} % Name
    {Position, University} % Relationship
    {} % Location
    {2016-2017} % Date(s)

%---------------------------------------------------------
  \cvhonor
    {Undergraduate Name} % Name
    {Position, University} % Relationship
    {} % Location
    {2014-2015} % Date(s)

%---------------------------------------------------------
  \cvhonor
    {Undergraduate Name} % Name
    {Position, University} % Relationship
    {} % Location
    {2014-2015} % Date(s)

%---------------------------------------------------------
\end{cvhonors}
\cvsection{Skills}
\vspace{4mm} 

\begin{cvskills}
  \skill
    {\textbf{Language}}
    {\textbf{German} {\cvskill{native}{1}{lightblue}} \textbf{English} {\cvskill{C2} {1}{lightblue}} \textbf{Spanish} {\cvskill{A2} {0.3}{lightblue}} \textbf{Chinese} {\cvskill{A1} {0.1}{lightblue}}} % Information
    \\
  \skill
    {\textbf{Programming}}
    {Python, C++, C, ROS 1, ROS 2, MATLAB, LaTeX\linebreak Linux, Git, Docker, Qt for Python, PX4 Autopilot, MAVSDK, MAVLink}
    \\
    \skill
    {\textbf{Tools}}
    {Siemens NX, Notion, Microsoft Office Suite, Adobe Photoshop \& Lightroom, Adobe Premiere Pro, WordPress}
    \\
\end{cvskills}

\cvsection{Interests \& Hobbies}

\vspace{3mm} 
\begin{cvhobbies}
    \cvhobby
    {since 2021}
    {Experienced (FPV) quadcopter pilot}
    {Cinematic \& freestyle FPV}
    
    \cvhobby
    {since 2019}
    {Calisthenics and weight training}
    {Fitness}
    
    \cvhobby
    {since 2018}
    {Multi-day backcountry experience}
    {Backpacking}
    
    \cvhobby
    {since 2012}
    {Medium-distance offroad running}
    {Cross-country}
    
    \cvhobby
    {since 2012}
    {Various tours and competitions}
    {Enduro mountain biking}
    
    \cvhobby
    {since 2010}
    {Portfolio: \href{https://pbreuer.de}{pbreuer.de}}
    {Photography}
    
    \cvhobby
    {since 2010}
    {YouTube: \href{https://www.youtube.com/channel/UCpW5zVz6mNaLRWC8l-Tq31A}{Peter Breuer}}
    {Filmmaking}
    
    \cvhobby
    {since 2007}
    {Former student at Musikschule Leipzig}
    {Piano}
\end{cvhobbies}


%-------------------------------------------------------------------------------
\end{document}
